\problem{设$f:\N\rightarrow\N$,证明:$f$可以作为配对函数的左函数当且仅当$\forall i\in\N, |\{z\in\N:f(z)=i\}|=\aleph_0$}
\begin{pf} \rm 
    % 该问题的核心是:相同的二元编码,只能唯一地解码为一对自然数分量,若可以编码任意的自然数对,那么在一个分量确定的前提下,另一个分量任取,一定能得到$\aleph_0$量级的编码.
设$Z_{x=i}=\{z\in\N:f(z)=i\}$. 若存在配对函数,设为$\pg(x,y)$,右函数设为$g(z)$. 

\paragraph{$\Rightarrow$}\quad\quad 根据可数选择公理,只需证明$\forall i, Z_{x=i}$是无限的. 假设对于某个$i$,$Z_{x=i}$有限,取$Y_{x=i}=\{j|g(z)=j\wedge z\in Z_{x=i}\}$,可知
$Y_{x=i}$也是有限的. 取任意$y\in \N-Y_{x=i}$,根据配对函数定义,$f(pg(i, y))=i$,这意味着$pg(i,y)=z\in Z_{x=i}$,这意味着$y\in Y_{x=i}$,矛盾.

\paragraph{$\Leftarrow$}\quad\quad 此时,对于任意的$i$,$Z_{x=i}$可以与$\N$建立一个双射$h_{x=i}:\N\rightarrow Z_{x=i}$,其逆为$h_{x=i}^{-1}$. 此时定义$g$如下:

$$ g(z)=\left\{
    \begin{aligned}
    &h^{-1}_{x=i}(z)   & z\in Z_{x=i}, \\
    &0 & \otherwise \\
    \end{aligned}
\right.
$$

令$\pg(x,y)=h_{x=x}(y)$,显然,$\forall x,y.\;f(pg(x,y))=x \wedge g(pg(x,y))=y$,满足配对函数定义.

    \qed
\end{pf}
