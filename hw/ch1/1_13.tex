\problem{对于斐波那契函数$f$,证明
(1) $f\in\PRF$
(2) $f\in\EF$.}\label{1.13}
\begin{pf} \rm 

寻找原始递归的构造时,
需要借助配对函数,使返回值可以包含多个值,用以传递前两层的结果。

我们令$\{\pg, K, L\}$为$\PRF$的一个配对函数,构造$F$:

$$
\begin{aligned}
    F(0)&=\pg(1, 0) \\
    F(x+1)&= \pg(K(F(x))+L(F(x)), K(F(x)))\\
\end{aligned}.
$$

此时,$f(x) = K(F(x)), f(x\dotminus 1) = L(F(x))$. 因而$f\in\PRF$.

了解到斐波那契递归对应的原始问题:$f(x)$计算了长度为$x-1$的不包含连续$1$的二进制串数量. (两个子问题:串首位为$0$或首位为$10$).

该问题可以用初等函数以遍历形式表达,以说明$f\in\EF$:

$$
f(n) = \sum_{i=0}^{2^{n-1}-1}N\left[ \sum_{i=0}^{n-2} \neqt\left(\frac{\rs(i,2^j)}{2^{j-1}}\right) 
\neqt\left(\frac{\rs(i,2^{j+1})}{2^{j}}\right)
\right]
$$

该函数对范围内满足检查的自然数进行计数:检查每个自然数的每相邻两位不存在同时等于$1$的情况.
    \qed
\end{pf}
