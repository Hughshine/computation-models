\problem{证明命题3.14:设$R$是$\Lambda$上的二元关系,$M\in\mathrm{NF}_R$,则(1) 不存在$N\in\Lambda$使得$M\rightarrow_R N$; (2) $M\twoheadrightarrow_R N\Rightarrow M\equiv N$.
}
\begin{pf} \rm \;
    \begin{enumerate}
        \item 根据$R$范式定义,$M$不存在$R$可约子项,所以$M$必然无法进行一步规约.
        \item 若$M\not\equiv N$,则必然存在一个长度大于$1$的$R$规约序列,这意味着$M$必然可以进行一步规约,与(1)矛盾.
    \end{enumerate}
    \qed 
\end{pf}