\problem{证明引理3.39.
}
\begin{pf} \rm \;
\begin{enumerate}
    \item $\forall n\in\N. \mathrm{var}(n)=\#(v^{(n)}) = [0, n]$ 显然是递归函数.
    \item $\forall M,N\in\Lambda. \mathrm{app}(\#M, \#N)=\#(M N) = [1, [\#M, \#N]]$ 显然是递归函数.
    \item $\forall x\in\nabla, M\in\Lambda.\mathrm{abs}(\#x, \#M) = \#(\lambda x.M) = [2,[\#x, \#M]]$ 显然是递归函数.
    \item 对于$\#\lceil n\rceil$,尝试找到它的递归式:
    $$
    \begin{aligned}
        \#\lceil n+1 \rceil 
        &= \#(\lambda fx.f^{n+1}x) \\
        &= [2, [\#f, \#(\lambda x.f^{n+1} x)]] \\
        &= [2, [\#f, [2, [\#x, \# f^{n+1} x]]]] \\
        &= [2, [\#f, [2, [\#x, [1, [\#f, \#f^nx]]]]]] \\
        &= [2, [\#f, [2, [\#x, [1, [\#f, (\pi_2)^4(\#\lceil n \rceil)]]]]]] \\
    \end{aligned}
    $$
    取$h(z) = [2, [\#f, [2, [\#x, [1, [\#f, (\pi_2)^4(z)]]]]]]$,则令:
    $$
    \begin{aligned}
        \mathrm{num}(0) &= \#\lceil 0 \rceil \\
        \mathrm{num}(n+1) &= h(num(n)).
    \end{aligned}
    $$
    显然$\mathrm{num}(n) = \#\lceil n \rceil$ 且 $\mathrm{num}\in\PRF$.
\end{enumerate}
    \qed
\end{pf}